\clearpage
\addcontentsline{toc}{chapter}{Abstract}

\begin{abstract}

The inherent problem of solar-powered base stations (BSs) will be tackled in this thesis, i.e., the problem that the energy generation of the photovoltaic (PV) cells does not match the energy consumption of the BS in time, which results in energy being wasted. 

In Chapter \ref{Introduction}, a comprehensive literature review is given. 

In Chapter \ref{Chapter_1}, the orientation angles of $N$ PV cells powering one BS are jointly optimized to improve the match between the two profiles on a daily timescale. The energy generation profiles of randomly inclined and oriented PV cells are analytically derived based on the Reindl model. The energy drawn per day from the main grid by the BS given its energy consumption profile is used as the performance metric to determine the optimal set of orientation angels. 
The main results are that deploying one PV cell (or several PV cells) with the (same) optimized orientation angle is recommended for BSs with an energy consumption profile that has one significant local maximum between sunrise and sunset. 
Deploying two PV cells (or two equal-sized groups of PV cells) where the two orientation angles (of the two groups) are jointly optimized is recommended for BSs with an energy consumption profile that has significant local maxima in the morning as well as in the afternoon or with a constant energy consumption profile.
 
In Chapter \ref{Chapter_2}, a battery model is added to the system model. The battery model is based on a Markov chain. The effects of different battery capacities on the optimal PV cell orientation angle are investigated. It is shown that PV cell orientation angle optimization should be done for BSs deployed with small batteries.

In Chapter \ref{Chapter_3}, the system model is extended to a multi-cell cellular network and a mixed-integer linear programming problem is developed to determine how energy harvesters with anti-correlated energy generation profiles should be deployed to every BS.

In Chapter \ref{conclusion}, the thesis is concluded.

\end{abstract}

