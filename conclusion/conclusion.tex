\clearpage
\chapter{Conclusion and Future Work\label{conclusion}}



\section{Summary of Thesis Achievements}

Chapter \ref{Chapter_1} focused on matching the energy generation profile of PV cells with the energy consumption profile of a BS in time. The orientation angles of $N$ PV cells powering one BS were jointly optimized to improve the match between the two profiles. The proposed optimization algorithm only needs to be run a single time offline and the obtained optimal angles can be used for all solar-powered BSs with similar geographic locations and energy consumption profiles.
The energy generation profile of any randomly inclined and oriented PV cell were analytically derived from the irradiance values received at a horizontally-mounted PV cell at the same location. The thesis identified and discussed analytically to what extent the orientation angle $\theta$ shifts the energy generation profile away from noon if the PV cells are not south-oriented ($\theta^*=0^{\mathrm{o}} $).
The energy drawn per day from the main grid by the BS given its energy consumption profile was used as the performance metric to determine the optimal set of orientation angels. To evaluate the effectiveness of the proposed orientation angle optimization, three different types of BS energy consumption profiles were investigated: constant traffic load profiles, business-area traffic load profiles, and residential-area traffic load
profiles.

The main results are that the system performance ($\Delta_1 > 0$) can be increased significantly by deploying one PV cell with optimal orientation angel $\theta_1^*$ (or several PV cells with the same orientation angle $\theta_1^*$) if the energy generation of the PV cell is slightly smaller ($G<C$), is slightly greater ($G>C$), or is significantly greater ($G>>C$) than the energy consumption of the BS. This is caused by the ability to shift the energy generation peak from noon towards the most significant local maximum between sunrise and sunset of the energy consumption profile. 
Furthermore, the system performance ($\Delta_2 > 0$) can be further increased by deploying two PV cells with jointly optimized orientation angles $\theta_1^*$ and $\theta_2^*$ (or several PV cells where half of them are deployed with $\theta_1^*$ and the other half with $\theta_2^*$) if a constant energy consumption profile or a consumption profile with significant local maxima in the morning as well as in the afternoon are given. This is caused by the ability to shift the energy generation peak from noon towards the morning with east-oriented PV cells, while the other west-oriented PV cells shift the energy generation peak towards the afternoon in the northern hemisphere. Because there are only two directions (morning and afternoon) that the energy can be shifted to, the system performance can not be further increased significantly by deploying more than 2 differently oriented PV cells. More than 2 differently oriented PV cells may even deteriorate the system performance ($\Delta_3 < 0$) in some scenarios.

Chapter \ref{Chapter_2} added a battery model to the system model. The battery model is based on a Markov chain. The PV cell's orientation angle optimization algorithm with Markov chain based battery model has a running time dependent on the squared number of energy states of the battery $S_{\max}$ and the time resolution $T$. The number of UEs served by the BS throughout the day $\overline{S_{\mathrm{UE}}}(\theta)$ was used as the performance metric to identify the optimal orientation angle. The accuracy of the proposed algorithm was verified by showing that simulation trials converge based on the law of large numbers to the output $\overline{S_{\mathrm{UE}}}(\theta)$ of the proposed algorithm. 
The effects of different battery capacities on the optimal PV cell orientation angle were investigated. 
Whereas BSs with small battery capacities significantly improved their performance by orientation angle optimization, BSs with large battery capacities should orient the PV cells towards the south. The importance of the PV cell orientation angle optimization was verified for a BS with small battery capacity ($b_{\max}=1$) located in a business-area in Greenwich (London, UK) in summer. Also PV cells are normally orientated to the south in Greenwich (London, UK), the proposed algorithm revealed that the optimal orientation angle is between $25^{\mathrm{o}}$ to $40 ^{\mathrm{o}}$ to the west. 





In the last chapter, the system model was extended to a multi-cell cellular network. The system model consisted of several BSs that were distributed in an area and some of them were connected by distribution lines to share the renewable energy among them. There were two different types of energy harvesters, which have anti-correlated energy generation profiles, available for deployment in the system model. Two PV cells that have significantly
different orientation angles, such as east-oriented and west-oriented PV cells, are an example
for energy harvesters with anti-correlated energy generation profiles. If energy harvesters with anti-correlated energy generation profiles are deployed at BSs that are connected by distribution lines, the power can be transmitted from surplus BSs to deficit BSs via the distribution lines. A mixed-integer linear programming problem (MILPP) was developed to determine how energy harvesters with anti-correlated energy generation profiles should be deployed to every BS to share the renewable power most efficiently in the cellular network. 
The MILPP can be run once during the cellular network planning. 
The MILPP takes into account the topology of the cellular network, i.e., whether or not a distribution line exists between a pair of
BSs. Hence, the proposed algorithm increases the probability that a BS is deployed with an energy harvester type that is anti-correlated to those deployed at its connected neighboring BSs. In addition, the shorter the distribution line between a pair of BSs, the more likely that these two BSs are deployed with anti-correlated energy harvesters, because the proposed algorithm takes into account the distance-dependent power loss in the distribution lines. 
The renewable power that can be transmitted from the surplus BSs to the deficit BSs in the cellular network was on average around 40\% higher with the proposed optimization algorithm in comparison with randomly deploying anti-correlated energy harvesters to the BSs.




\section{Future Work and Applications}

In the age of internet, online tools, and publicly available data, future applications and optimization tools on basis of this thesis are likely to be implemented. There are already an increasing amount of geographic information systems (GISs) online available, which provide PV cell performance guides for various geographical areas free of charge, such as for the USA \cite{usa}, for Europe \cite{PVGIS}, and worldwide \cite{worldwide,worldwide2}. An online tool for PV cell orientation angle optimization can be easily implemented on basis of this thesis. Such an online tool would greatly increase the impact of this thesis and will be beneficial for mankind and the environment. For that purpose, the source code of Chapter \ref{Chapter_1} is given in the Appendix B of this thesis and is publicly available on GitHub \cite{DOBEN_GITHUB}.

This thesis derived the optimization methods necessary to optimize PV cells’ orientation angles and their deployment at BSs for many different scenarios. For that purpose, the methods presented in this thesis are kept general and normalized energy generation values and normalized energy consumption values are used. The advantage of generalization and normalization is that the derived optimization methods presented in this thesis can be applied to nearly any scenario in the real world. If the input parameters for a specific scenario in the real world, such as PV cell types, energy profiles, and cellular network topology, are given, the optimal orientation angles and deployment strategy can be derived with the methods presented in this thesis. The disadvantage of generalization and normalization is that the thesis did not focus on deriving a comprehensive practical worldwide guide for a specific input parameter set. Future extension to this work can be done by evaluation typical specific input parameter sets and deriving for each set a comprehensive practical worldwide guide. In addition, if the specific input parameter set is derived from a real world system, future work can be done by evaluating the performance of the real world system deployed with the optimal PV cells’ orientation angles and deployment strategy.

An interesting future extension is to increase the optimization horizon to an entire year. Optimizing the inclination angle of a PV cell is done on a yearly timescale. As a result, it is a method to shift the energy generation peak from a surplus season (e.g. summer) to a deficit season (e.g. winter). In contrast, optimizing the PV cell orientation angle is done on a daily timescale. As a result, it is a method to shift the energy generation peak from a surplus time (e.g. noon) to a deficit time (e.g. morning or afternoon). Nonetheless, most geographical locations have different seasonal energy generation profiles. Hence, the optimal PV cells’ orientation angles and deployment strategies are different in different seasons. Our derived optimization method can be applied to several days distributed throughout the year (e.g. one day in winter, one day in spring, one day in summer, and one day in autumn) and the average derived PV cells’ orientation angles and deployment strategy can be used for deployment. In spite of that strategy, many locations in the northern hemisphere might emphasis on improving the performance of the system during the winter season since this is usually the most energy sparse season. A higher weight can be given on the optimal PV cells’ orientation angles and deployment strategy in winter compared with the other seasons. Increasing the optimization horizon to an entire year and weighting the different seasons are an interesting extension to this thesis.

Furthermore, future work can jointly optimize the orientation and inclination angles of PV cells.



Some of the models in this thesis can be extended to model the real world more accurately in the future. For example, the battery model could include input/output efficiency coefficients since there is always a leakage when charging/discharging a battery. Another example, since the energy needed to serve one UE at the BS during one time step varies considerably dependent on the UEs distance to the BS, the requested service, and priority or interference issues with other UEs, the average amount of energy needed to serve an UE could be modeled more realistically in future extensions of this work.   




